\documentclass[a4paper,10pt]{scrartcl}
\usepackage[utf8]{inputenc}
\usepackage[T1]{fontenc}
\usepackage[naustrian]{babel}
\usepackage{lmodern}
\usepackage{graphicx}
\usepackage{hyperref}
\usepackage{multirow}

%\usepackage{tabularx}
%\usepackage{amsmath}

%opening
\title{HCI Meilenstein 4}
\subtitle{Elektronisches Curriculum}
\author{Team 1 \\Pascal Attwenger, Philipp Hiermann, Sandra Markhart}

\begin{document}

\maketitle

\section{Einleitung}

Im Verlauf der letzten Wochen wurde in Gruppenarbeit ein neues Interface für die Prüfungsleistungen und das Mittelungsblatt (``eCurriculum'') entwickelt.
Ziel dieser Studie ist es zu überprüfen, ob unser neues Interface wirklich besser zu bedienen ist als das alte Interface.

x

\section{Beschreibung des Studienaufbaus} 

Die Studie vergleicht unser neues Interface mit dem altem Interface (univis/Mitteilungsblatt), in Folge dessen, sind alle Hypothesen auf diese
Interfaces anzuwenden und die Aufgaben werden immer jeweils mit beiden Interfaces durchgeführt.

\subsection{Hypothesen}

\begin{description}
 \item[Alternativhypothese 1: ] User sind mit unserem neu erstellten Interface zufriedener als mit dem alten Interface. 
\end{description}

\begin{description}
 \item[Alternativhypothese 2: ] User können mit unserem neu erstellten Interface Aufgaben besser lösen als mit dem alten Interface. 
\end{description}

\subsection{Testbare Aufgaben}

\begin{itemize}
 \item Welche Lehrveranstaltungen können im Modul ``Modul VMI Vertiefung Medieninformatik'' absolviert werden? (Mitteilungsblatt)
 \item Wieviele ECTS bringt die UE Arbeitstechniken Multimediajournalismus? (Mitteilungsblatt)
 %\item Wieviele ECTS habe ich in meinem Studium bereits absolviert?
 %\item Wieviele Module habe ich bereits in meinem Studium abgeschlossen?
 \item Welche Note habe ich in der Übung ``UE Technische Grundlagen und Systemsoftware''? (univis)
\end{itemize}

\subsection{Variablen}

\subsubsection{Performanzmetriken}

\begin{itemize}
 \item Zeit in Sekunden
 \item Anzahl der Klicks bei den univis-Aufgaben
\end{itemize}

\subsubsection{Subjektives Nutzerfeedback}

Ein AttrakDiff-Fragebogen, bei dem folgende Eigenschaften mit 7 verschiedenen Möglichkeiten zu bewerten sind:

\begin{itemize}
\item einfach - kompliziert 
\item menschlich - technisch
\item verständlich - unverständlich
\item übersichtlich - unübersichtlich 
\item innovativ - konservativ
\item kreativ - phantasielos
\item schön - hässlich
\item fröhlich - deprimierend 
\end{itemize}

\begin{description}
 \item[Datei auf cewebs:]Meilenstein 4 - Team 1 - Fragebogen
\end{description}

\subsection{Testpersonen}

\begin{description}
 \item[Testperson 1:] weiblich, 47, noch nie mit Univis gearbeitet
 \item[Testperson 2:] männlich, 27, studiert Medieninformatik an der Uni Wien
\end{description}

\subsection{Methodik}



\section{Resultaten} 

\begin{center}
\begin{tabular}{r|r|r|r|r}
    & & Aufgabe 1 & Aufgabe 2 & Aufgabe 3 \\
    \hline
    \multirow{2}{*}{Testperson 1} & Univis & 1min 30s & 1min 29s & 2min 28s\\
    & eCurriculum & 10s & 51s & 19s\\
    \hline
    \multirow{2}{*}{Testperson 2} & Univis & 5s & 8s & 32s\\
    & eCurriculum & 10s & 15s & 16s\\
\end{tabular}
\end{center}

Stellen sie die Ergebnisse mittels deskriptiver Statistiken dar (tabellarisch und visuell). 
Optional: Führen sie statistische Signifikanztests durch, z.b. mittels t-test oder ANOVA (maximal +10 Bonuspunkte). 

\section{Diskussion}


\section{Schlussfolgerungen}


\section{Appendix}

\section{Reflexion über das Gesamtprojekt}

\subsection{Kernerkenntnisse und Designempfehlungen aus dem Projekt}

tabellarische Form des Studienplans ist gut angekommen, Mitteilungsblatt in elektronischer Form, Gliederung des Curriculums nach Semester anstatt der eigenartigen
Modulgruppen, direkte Anmeldung zu den Lehrveranstaltungen wenn diese noch nicht absolviert wurden, Studienfortschritt als Anzeige, we are motherfucking awesome

\subsection{Arbeitsverteilung, Kommunikation, Lernprozess und Zufriedenheit}


    Fassen Sie die Kern-Erkenntnisse ihres Projektes zusammen. Beschreiben sie insb. 3-5 konkrete Empfehlungen zu Designverbesserungen, 
    die sie aus ihrem Gesamtprojekt abgeleitet haben. Diese Zusammenfassung ist als "executive summary" für das UniVis- bzw. das Literacy-team gedacht, 
    bzw. fuer einen imaginären Kunden, falls sie ein eigenes Thema gewählt haben (max. 1 Seite Text + evtl. Screenshots).
    Kurze Beschreibung zu Arbeitsverteilung, Kommunikation, Lernprozess und Zufriedenheit (max 1 Seite). 



\end{document}
