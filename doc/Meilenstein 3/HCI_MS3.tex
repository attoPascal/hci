\documentclass[a4paper,10pt]{scrartcl}
\usepackage[utf8]{inputenc}
\usepackage[T1]{fontenc}
\usepackage[naustrian]{babel}
\usepackage{lmodern}
\usepackage{graphicx}
\usepackage{hyperref}
%\usepackage{tabularx}
%\usepackage{amsmath}

%opening
\title{HCI Meilenstein 3}
\subtitle{Elektronisches Curriculum}
\author{Team 1 \\Pascal Attwenger, Philipp Hiermann, Sandra Markhart}

\begin{document}

\maketitle

\section{Usability Test Aufgaben}

\subsection{Aufgabe 1}

\subsubsection*{Ziel:}
 
Der User soll herausfinden welche Fächer er im 1. Semester absolvieren muss und kann.
 
\subsubsection*{Beschreibung:}

Du fängst nächstes Semester an Medieninformatik zu studieren. Für welche Fächer sollst du dich anmelden?
 
\subsection{Aufgabe 2}

\subsubsection*{Ziel:}

Der User soll seine Note abfragen.

\subsubsection*{Beschreibung:}

Du hast vor 2 Wochen die Vorlesungsprüfung zu VO Netzwerktechnologien abgelegt. Hast du bereits eine Note erhalten? Wenn ja welche?

\subsection{Aufgabe 3}

\subsubsection*{Ziel:}

Der User soll seinen Notendurchschnitt herausfinden.

\subsubsection*{Beschreibung:}

Du studierst an der Universität Wien und willst wissen ob du dich für ein Leistungsstipendium eignest. Das Stipendium verlangt einen Notendurchschnitt von
1,8. Kannst du ein Leistungsstipendium erhalten?


\section{Interviewleitfaden}


\section{Abschlussinterview}


\section{Bericht}

\subsection{Aufgabe 1}

\subsection{Aufgabe 2}

\subsection{Aufgabe 3}

\subsection{Gesamteinschätzung}

\subsection{Verbesserungsvorschläge}



\end{document}
